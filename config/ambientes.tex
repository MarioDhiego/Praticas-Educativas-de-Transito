% Definição de ambientes exercício, solução, definição, exemplo, demonstração, etc.

\usepackage{mdframed}

\usepackage{environ} % usado nas definições \MakeDiscussionTopSecret, \MakeDiscussionsPublic, etc.

% Exercício e solução (pacote Exsheets):

\usepackage{exsheets} %[2015/02/09]

% Configurações do pacote Exsheets:

\SetupExSheets{
  headings          = block-subtitle ,
  subtitle-format   = \sc ,
  counter-within    = {chapter} , % Contar dentro dos capítulos.
  counter-format    = ch.qu[1] , % Formato 1.1, 1.2, 1.3, etc.
  label-format      = ch.qu[1] , % Formato 1.1, 1.2, 1.3, etc.
  headings-format   = \bfseries ,
  question/pre-hook = \needspace{0.3\textheight}\vspace{1ex} ,
  question/post-hook = \vspace{1ex} ,
  question/pre-body-hook = {\vspace{1ex} \mdframed[backgroundcolor=verde_UnB!10, linewidth=1pt, innermargin=+0cm, outermargin=+0cm]},
  question/post-body-hook = \endmdframed ,
  solution/pre-hook = \needspace{0.3\textheight}\vspace{1ex},
  solution/post-hook = \vspace{1ex},
  solution/pre-body-hook = \vspace{1ex},
  solution/print    = false
}

\renewcommand\thequestion{\thechapter.\arabic{question}} % Formato 1.1, 1.2, 1.3, etc., ao usar \ref{}:

% Ambiente "Definição":

\mdfdefinestyle{definitionSty}{backgroundcolor=gray!10, linecolor=verde_UnB, linewidth=0pt, innerleftmargin=3ex, innerrightmargin=3ex, innertopmargin=1ex, innerbottommargin=1ex, innermargin =0, outermargin =0, needspace={0.2\textheight}}
\newcounter{definitionCounter}[chapter]
\numberwithin{definitionCounter}{chapter}
\theoremstyle{definition}
\newmdtheoremenv[style=definitionSty]{definition}[definitionCounter]{Defini\c{c}\~{a}o}

% Ambiente "Teorema":

\mdfdefinestyle{theoremSty}{backgroundcolor=verde_UnB!10, linecolor=verde_UnB, linewidth=0pt, innerleftmargin=3ex, innerrightmargin=3ex, innertopmargin=1ex, innerbottommargin=1ex, innermargin=0, outermargin=0, needspace={0.2\textheight}}
\newcounter{theoremCounter}[chapter] 
\numberwithin{theoremCounter}{chapter}
\newmdtheoremenv[style=theoremSty]{theorem}[theoremCounter]{Teorema}

% Ambiente "Demonstração":

\mdfdefinestyle{demonstrationSty}{backgroundcolor=gray!10, linecolor=verde_UnB, linewidth=0pt, innerleftmargin=3ex, innerrightmargin=3ex, innertopmargin=1ex, innerbottommargin=1ex, innermargin =0cm, outermargin =0cm, needspace={0.2\textheight}}
\newcounter{demonstrationCounter}[chapter]
\numberwithin{demonstrationCounter}{chapter}
\newmdtheoremenv[style=demonstrationSty]{demonstration}[demonstrationCounter]{Demonstra\c{c}\~{a}o}

% Ambiente "Axioma":

\mdfdefinestyle{axiomSty}{backgroundcolor=verde_UnB!10, linecolor=verde_UnB, linewidth=1pt, innerleftmargin=3ex, innerrightmargin=3ex, innertopmargin=1ex, innerbottommargin=1ex, innermargin=0, outermargin=0, needspace={0.1\textheight}}
\newcounter{axiomCounter}[chapter] 
\numberwithin{axiomCounter}{chapter}
\newmdtheoremenv[style=axiomSty]{axiom}[axiomCounter]{Axioma}

% Ambiente "Exemplo":

\mdfdefinestyle{exampleSty}{backgroundcolor=gray!10, linecolor=verde_UnB, linewidth=0pt, innerleftmargin=3ex, innerrightmargin=3ex, innertopmargin=1ex, innerbottommargin=1ex,  innermargin =+0cm, outermargin =+0cm, needspace={0.2\textheight}}
\newcounter{exampleCounter}[chapter]
\numberwithin{exampleCounter}{chapter}
\newmdtheoremenv[style=exampleSty]{example}[exampleCounter]{Exemplo}

% Ambiente "Problema":

\mdfdefinestyle{problemSty}{backgroundcolor=gray!10, linecolor=black, linewidth=1pt, innerleftmargin=3ex, innerrightmargin=3ex, innertopmargin=1ex, innerbottommargin=1ex, innermargin =+0cm, outermargin =+0cm, needspace={0.15\textheight}}
\newcounter{problemCounter}[chapter]
\numberwithin{problemCounter}{chapter}
\newmdtheoremenv[style=problemSty]{problem}[problemCounter]{Problema}

% Ambiente "Nota":

\mdfdefinestyle{noteSty}{backgroundcolor=white, font=\small, fontcolor=verde_UnB, linecolor=verde_UnB, linewidth=1pt, innerleftmargin=3ex, innerrightmargin=3ex, innertopmargin=2ex, innerbottommargin=2ex, innermargin =+0cm, outermargin =+0cm, needspace={0.1\textheight}}
\newcounter{noteCounter}[chapter]
\numberwithin{noteCounter}{chapter}
\newmdtheoremenv[style=noteSty]{note}[noteCounter]{Nota}

% Ambiente "Discussão" (texto complementar):

\mdfdefinestyle{discussionSty}{backgroundcolor=white, fontcolor=verde_UnB, linecolor=verde_UnB, linewidth=0pt, innerleftmargin=3ex, innerrightmargin=3ex, innertopmargin=3ex, innerbottommargin=3ex, innermargin=+0cm, outermargin =+0cm, needspace={0.25\textheight}}
\newcounter{discussionCounter}%[chapter]
%\numberwithin{discussionCounter}{chapter}
\newmdtheoremenv[style=discussionSty]{discussion}[discussionCounter]{Texto Complementar}

% Ambiente "Dedicatória" (baseado em https://tex.stackexchange.com/a/167529/91816):

\newenvironment{dedication}
  {
    \itshape             % the text is in italics
    \raggedleft          % flush to the right margin
  }
  {
    \par % end the paragraph
  }
  % space at bottom is three times that at the top
  
% Citação com autor:

\renewenvironment{quotation}
  {\small\list{}{\rightmargin=0.1cm \leftmargin=4cm \topsep=1.5\baselineskip}%
   \item\relax}
  {\endlist}

\def\signed #1{{\leavevmode\unskip\nobreak\newline
  \hbox{}\nobreak #1 %
  \hfil\parfillskip=0pt \finalhyphendemerits=0 \endgraf}}

\newsavebox\mybox
\newenvironment{citacao}[1]
  {\savebox\mybox{#1}\begin{quotation}}
  {\signed{\usebox\mybox}\end{quotation}}
