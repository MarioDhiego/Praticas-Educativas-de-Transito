% Modelo de livro da série Ensino de Graduação da Editora UnB
% Autor: Leonardo Luiz e Castro

% Compile usando LuaLaTeX.

\documentclass{book}

 

% Chamadas e configurações de pacotes estão na pasta config:

% O comando abaixo pode ser necessário em documentos que usam versões antigas do pacote "fontspec". Se for começar um novo documento, deixe esse linha comentada e use caracteres Unicode.
% \usepackage[utf8]{luainputenc}

% Tamanho do papel e das margens: 

\usepackage[paperheight=230mm,paperwidth=160mm,inner=20mm,outer=14mm,top=24.5mm,bottom=21mm,pdftex]{geometry}

% Suporte a idiomas:

\usepackage[greek,brazilian]{babel}

% Para usar as variáveis \theauthor, \thetitle e \thedate (autor, título e data).

\usepackage{titling}

% Título de capítulo elegante:

\usepackage[Lenny]{fncychap}

% Pacote para estilo elegante:

\usepackage{fancyhdr}

% Definição de novas cores:

\usepackage{xcolor}

\definecolor{verde}{rgb}{0.2, 0.50, 0.25}
\definecolor{verde_UnB}{cmyk}{1,0,1,0.2}
\definecolor{cinza_UnB}{rgb}{0.6,0.6,0.6} % https://www.ginifab.com/feeds/pms/color_picker_from_image.php


% Para mudar cor de títulos (https://tex.stackexchange.com/a/75670/91816):

\usepackage{sectsty}

\chapterfont{\color{verde_UnB}}  % sets colour of chapters
\sectionfont{\color{verde_UnB}}  % sets colour of sections

% Pacote para definir fonte de qualquer tamanho:

\usepackage{fix-cm}

%% Pacote para correta separação silábica de palavras com hífen;
%% hífens devem ser escritos como \Hyphdash ou \-/ (se opção shortcuts estiver ativa)
%% no texto; por exemplo cana\-/de\-/açúcar seria separada como
%% cana-de- numa linha e -açúcar na linha seguinte
%% repetindo o hífen (um hífen é da separação e outro é da palavra):

\usepackage[shortcuts]{extdash}

%% Definindo espacamento:

\usepackage{setspace} 
\singlespacing

%% Enumerar itens:

\usepackage{enumitem}

%% Pacote para criar caixas:

\usepackage{pbox} %  que limitam a largura de textos em células,
%\usepackage{minibox} % que têm largura arbitrária.

%% Definição das dimensões do texto:

%\setlength{\textwidth}{16cm}
%\setlength{\textheight}{22cm}
%\setlength{\headheight}{1cm}
%\setlength{\footheight}{1cm}

\usepackage[title,titletoc]{appendix}

%% Passando títulos para o português:

\renewcommand{\chaptername}{Capítulo}
\renewcommand{\bibname}{Referências}
\renewcommand{\appendixname}{Apêndice}
\renewcommand{\indexname}{Índice Remissivo}
\renewcommand{\contentsname}{\bf\color{verde_UnB}Sumário}
\renewcommand{\tablename}{Tabela}
\renewcommand{\figurename}{\bf Fig.}
\renewcommand{\sin}{sen}
\def\listoffiguresname{Lista de Figuras}
\def\listoftablesname{Lista de Tabelas}

%% Definindo estilo de pagina:

\pagestyle{fancy}

% PACOTES DE TABELAS:

%% Espaçamento de tabelas ajustado:

\usepackage{booktabs}

%% Pacotes de linhas e colunas multiplas:

\usepackage{multirow}
\usepackage{multicol}

%% Pacotes de tabela e tabulação:

%%% Tabela com caixas centralizada:

\usepackage{array}
\newcolumntype{P}[1]{>{\centering\arraybackslash}p{#1}}

%%% Tabela em múltiplas páginas:

\usepackage{longtable}

%%% Tabela colorida:

\usepackage{tabu}
\usepackage{colortbl}

%%% Mudando cor das linhas de todas as tabelas:

\makeatletter
\renewenvironment{table}
     {\@float{table}\taburulecolor{verde_UnB}\arrayrulecolor{verde_UnB}}
     {\end@float}
\makeatother

%%% Tabela com linhas pontilhadas:

\usepackage{arydshln}

% PACOTES MATEMÁTICOS:

%% Pacote para não-itálico em ambiente de fórmulas:
%\usepackage[]{mathastext}

%% Pacotes matemáticos:

\usepackage{amsmath}
\usepackage{mathtools}

%%% Numera apenas equações usadas:

\mathtoolsset{showonlyrefs=true}

%%% Declarando barras para valor absoluto e norma:

\DeclarePairedDelimiter\abs{\lvert}{\rvert}%
\DeclarePairedDelimiter\norm{\lVert}{\rVert}%

%% Nome de equações ao lado com comando "eqname":

\newcommand{\eqname}[1]{\tag*{\llap{#1}}}

%% Fontes tipográficas:

\usepackage{fontspec}
%\usepackage{libertineotf}

%%% Seleção da fonte UnB Pro:

\setmainfont{UnB-Pro}[
  Path = fonts/UnB_Pro_v1.0/,
  Extension=.otf,
  UprightFont=*_Regular,
  ItalicFont=*_Italic,
  BoldFont=*_Bold,
  BoldItalicFont=*_Bold-Italic,
]

\setmonofont{Libertinus Mono}

%%% Para incluir o comando \url{}:

\usepackage[breaklinks=true]{hyperref}

%% Configurações ABNTeX:

\usepackage[hyphens,alf,abnt-and-type=e,abnt-last-names=bibtex,abnt-etal-cite=1,abnt-etal-list=1]{abntex2cite}


% Recuo no primeiro parágrafo:

\usepackage{indentfirst}

% Mudando recuo de parágrafo e tamanho da fonte:

\setlength{\parindent}{7mm}
\fontsize{11pt}{13.2pt}\selectfont

% Usado para evitar linhas órfãs:

\usepackage{needspace}

% Mudar espaçamento entre número e título de seção:

%% \def\l@figure{\@dottedtocline{1}{1.5em}{2.5em}}

\usepackage{tocloft}
\setlength{\cftfignumwidth}{2.55em} 

% Para escrever "Apêndices" e "Anexos":

%\usepackage[titletoc]{appendix}

% Elimina recuo nas notas de rodapé:

\usepackage[hang,flushmargin]{footmisc}

% Primeira página vazia em cada capítulo:

\makeatletter
\renewcommand\chapter{\if@openright\cleardoublepage\else\clearpage\fi
                    \thispagestyle{empty}% original style: plain
                    \global\@topnum\z@
                    \@afterindentfalse
                    \secdef\@chapter\@schapter}
\makeatother

% Diminui chance de linhas órfãs e viúvas:

\clubpenalty1000000
\widowpenalty1000000

% Comando para mudar a fonte de citação literal de código (verbatim):

\makeatletter
\newcommand{\verbatimfont}[1]{\def\verbatim@font{#1}}%
\makeatother

% Pacote para referir a capítulos pelo nome:
\usepackage{nameref}
\makeatletter
\newcommand*{\currentname}{\@currentlabelname}
\makeatother

% Definições (alternativas a nameref) para se referir a capítulos e seções pelo nome:
\let\Chaptermark\chaptermark
\def\chaptermark#1{\def\Chaptername{#1}\Chaptermark{#1}}
\let\Sectionmark\sectionmark
\def\sectionmark#1{\def\Sectionname{#1}\Sectionmark{#1}}
\let\Subsectionmark\subsectionmark
\def\subsectionmark#1{\def\Subsectionname{#1}\Subsectionmark{#1}}
\let\Subsubsectionmark\subsubsectionmark
\def\subsubsectionmark#1{\def\Subsubsectionname{#1}\Subsubsectionmark{#1}}

% Pacote para usar "clearpage" sem finalizar página, apenas descarregar todos os elementos já inseridos:

\usepackage{afterpage}

% Pacote para que páginas em branco não mostrem cabeçalhos nem rodapés:

\usepackage{emptypage}

% Pacote para índice Remissivo (a ser impresso por \printindex):

\usepackage{makeidx}
\makeindex

% Pacotes auxiliares no design da capa:

\usepackage{xcoffins}

%% Pacotes para usar o pacote Metapost para fazer graficos e diagramas:

\usepackage{luamplib}
\everymplib{input mpcolornames; beginfig(1);}
\everyendmplib{endfig;}

%% Pacotes para diagramas, desenhos, gráficos:

\usepackage{tikz}
\usetikzlibrary{backgrounds,patterns}

\usepackage{pgfplots}
\pgfplotsset{compat=1.12}

%% Pacote para ferramentas gráficas:

\usepackage{graphics}
\usepackage{graphicx} % Sem esse, alguns includegraphics não funcionam.

\usepackage{threeparttable} % Para alinhar figuras e legendas com "measuredfigure".

\usepackage{wrapfig} % Figuras ao lado do texto.

\usepackage{picinpar}% Alteranativa para figuras ao lado de texto. http://ctan.org/pkg/picinpar

\usepackage[font={small}, margin=0cm, justification=centering]{caption} % Legendas com fonte pequena.

% Comando para adicionar fonte de figuras e semelhantes:

\newcommand{\source}[1]{\captionsetup{singlelinecheck=false,justification=justified}\caption*{\footnotesize \noindent Fonte: {#1}}}
% Tamanhos de fontes matemáticas em relação à fonte do texto:
%% {tamanho do texto} {matemática} {matemática script} {matemática scriptscript}

\DeclareMathSizes{10}{10}{6}{4}
\DeclareMathSizes{9}{9}{5}{3}

\usepackage{amsthm} % deve ser chamado antes de mdframed conforme dito em http://tex.stackexchange.com/questions/283763/why-dont-i-get-non-italic-normal-font-inside-theorem-environment-using-newmdth

% Seleção de fontes de equações:

\usepackage[math-style=french]{unicode-math} % não funciona se compilar com pdflatex, deve-se usar xelatex ou lualatex

\usepackage{yfonts} % para usar caracteres góticos para algumas variaveis

% Pacote para maior controle de fluxo de figuras (usado para opção [H] que fixa a posição das figuras):

\usepackage{float}

% Pacote usado para impedir elementos flutuantes (figuras, tabelas, etc.) de aparecerem em seção errada:

\usepackage[section]{placeins}

% Pacote inserido para fazer o símbolo de grau:

\usepackage{gensymb}

% Pacote inserido para fazer símbolos de circuito elétrico:

\usepackage{marvosym}

% Definição de variáveis e unidades:

\newcommand{\angstrom}{\text{\normalfont\AA}}
\newcommand{\pol}{\ensuremath{pol}}
\newcommand{\cm}{\ensuremath{cm}}
\newcommand{\km}{\ensuremath{km}}
\newcommand{\hm}{\ensuremath{hm}}
\newcommand{\dam}{\ensuremath{dam}}
\newcommand{\dm}{\ensuremath{dm}}
\newcommand{\mol}{\ensuremath{mol}}
\newcommand{\mi}{\ensuremath{mi}}
\newcommand{\h}{\ensuremath{h}}
\newcommand{\s}{\ensuremath{s}}
\newcommand{\Min}{\ensuremath{min}}
\newcommand{\Pa}{\ensuremath{Pa}}
\newcommand{\atm}{\ensuremath{atm}}
\newcommand{\mmHg}{\ensuremath{mmHg}}
\newcommand{\BarP}{\ensuremath{Bar}}
\newcommand{\PsiP}{\ensuremath{PSI}}
\newcommand{\lb}{\ensuremath{lb}}
\newcommand{\N}{\ensuremath{N}}
\newcommand{\kg}{\ensuremath{kg}}
\newcommand{\kgf}{\ensuremath{kgf}}
\newcommand{\are}{\ensuremath{a}}
\newcommand{\litro}{\ensuremath{\ell}}
\newcommand{\g}{\ensuremath{g}}

% Símbolos de diferencial:

\def\D{\mathrm{d}} %% diferencial - comando \D{}
\def\DI{{\delta}} % diferencial inexata ``delta''

\newcommand*\diff{\mathop{}\!\mathrm{d}}
\newcommand*\Diff[1]{\mathop{}\!\mathrm{d^#1}}

% Evita que equações extrapolem a margem (e permite outros recursos tipográficos avançados):

\usepackage{microtype}

%% Simbolos matemáticos extra:

\DeclareMathSymbol{\Omega}{\mathalpha}{letters}{"0A}
\DeclareMathSymbol{\varOmega}{\mathalpha}{operators}{"0A}
\providecommand*{\upOmega}{\varOmega} % for siunitx

%% Pacote para uso de unidades SI com distância padronizada entre valor e unidade:

\usepackage{siunitx}
\sisetup{%
      binary-units=true,
      group-separator={.},
      group-digits=integer,
      load-configurations=abbreviations,
      load=addn,
      per-mode=fraction,
      output-decimal-marker={,},
      range-phrase= --,
      separate-uncertainty=true,
      math-ohm = \ensuremath{\upOmega}, % senão \ohm não funciona
      text-ohm = Ω,  % senão \ohm não funciona
    }

\DeclareSIUnit\milha{mi}
\DeclareSIUnit\polegada{pol}
\DeclareSIUnit\alqueire{alqueire}
\DeclareSIUnit\inch{in}
\DeclareSIUnit\foot{ft}
\DeclareSIUnit\kgf{kgf}
\DeclareSIUnit\lbf{lbf}
\DeclareSIUnit\pound{lb}
\DeclareSIUnit\rev{rev}
\DeclareSIUnit\rpm{rpm}
\DeclareSIUnit\pkt{PKT}
\DeclareSIUnit{\calorie}{cal}
\DeclareSIUnit{\cal}{cal}
\DeclareSIUnit{\Cal}{Cal}
\DeclareSIUnit{\Calorie}{\kilo\calorie}
\DeclareSIUnit{\fahrenheit}{\degree F}

\DeclareSIUnit{\nothing}{\relax} % usado para mostrar prefixos sem unidade

% Equações químicas:

\usepackage[version=4]{mhchem}

% Definição de ambientes exercício, solução, definição, exemplo, demonstração, etc.

\usepackage{mdframed}

\usepackage{environ} % usado nas definições \MakeDiscussionTopSecret, \MakeDiscussionsPublic, etc.

% Exercício e solução (pacote Exsheets):

\usepackage{exsheets} %[2015/02/09]

% Configurações do pacote Exsheets:

\SetupExSheets{
  headings          = block-subtitle ,
  subtitle-format   = \sc ,
  counter-within    = {chapter} , % Contar dentro dos capítulos.
  counter-format    = ch.qu[1] , % Formato 1.1, 1.2, 1.3, etc.
  label-format      = ch.qu[1] , % Formato 1.1, 1.2, 1.3, etc.
  headings-format   = \bfseries ,
  question/pre-hook = \needspace{0.3\textheight}\vspace{1ex} ,
  question/post-hook = \vspace{1ex} ,
  question/pre-body-hook = {\vspace{1ex} \mdframed[backgroundcolor=verde_UnB!10, linewidth=1pt, innermargin=+0cm, outermargin=+0cm]},
  question/post-body-hook = \endmdframed ,
  solution/pre-hook = \needspace{0.3\textheight}\vspace{1ex},
  solution/post-hook = \vspace{1ex},
  solution/pre-body-hook = \vspace{1ex},
  solution/print    = false
}

\renewcommand\thequestion{\thechapter.\arabic{question}} % Formato 1.1, 1.2, 1.3, etc., ao usar \ref{}:

% Ambiente "Definição":

\mdfdefinestyle{definitionSty}{backgroundcolor=gray!10, linecolor=verde_UnB, linewidth=0pt, innerleftmargin=3ex, innerrightmargin=3ex, innertopmargin=1ex, innerbottommargin=1ex, innermargin =0, outermargin =0, needspace={0.2\textheight}}
\newcounter{definitionCounter}[chapter]
\numberwithin{definitionCounter}{chapter}
\theoremstyle{definition}
\newmdtheoremenv[style=definitionSty]{definition}[definitionCounter]{Defini\c{c}\~{a}o}

% Ambiente "Teorema":

\mdfdefinestyle{theoremSty}{backgroundcolor=verde_UnB!10, linecolor=verde_UnB, linewidth=0pt, innerleftmargin=3ex, innerrightmargin=3ex, innertopmargin=1ex, innerbottommargin=1ex, innermargin=0, outermargin=0, needspace={0.2\textheight}}
\newcounter{theoremCounter}[chapter] 
\numberwithin{theoremCounter}{chapter}
\newmdtheoremenv[style=theoremSty]{theorem}[theoremCounter]{Teorema}

% Ambiente "Demonstração":

\mdfdefinestyle{demonstrationSty}{backgroundcolor=gray!10, linecolor=verde_UnB, linewidth=0pt, innerleftmargin=3ex, innerrightmargin=3ex, innertopmargin=1ex, innerbottommargin=1ex, innermargin =0cm, outermargin =0cm, needspace={0.2\textheight}}
\newcounter{demonstrationCounter}[chapter]
\numberwithin{demonstrationCounter}{chapter}
\newmdtheoremenv[style=demonstrationSty]{demonstration}[demonstrationCounter]{Demonstra\c{c}\~{a}o}

% Ambiente "Axioma":

\mdfdefinestyle{axiomSty}{backgroundcolor=verde_UnB!10, linecolor=verde_UnB, linewidth=1pt, innerleftmargin=3ex, innerrightmargin=3ex, innertopmargin=1ex, innerbottommargin=1ex, innermargin=0, outermargin=0, needspace={0.1\textheight}}
\newcounter{axiomCounter}[chapter] 
\numberwithin{axiomCounter}{chapter}
\newmdtheoremenv[style=axiomSty]{axiom}[axiomCounter]{Axioma}

% Ambiente "Exemplo":

\mdfdefinestyle{exampleSty}{backgroundcolor=gray!10, linecolor=verde_UnB, linewidth=0pt, innerleftmargin=3ex, innerrightmargin=3ex, innertopmargin=1ex, innerbottommargin=1ex,  innermargin =+0cm, outermargin =+0cm, needspace={0.2\textheight}}
\newcounter{exampleCounter}[chapter]
\numberwithin{exampleCounter}{chapter}
\newmdtheoremenv[style=exampleSty]{example}[exampleCounter]{Exemplo}

% Ambiente "Problema":

\mdfdefinestyle{problemSty}{backgroundcolor=gray!10, linecolor=black, linewidth=1pt, innerleftmargin=3ex, innerrightmargin=3ex, innertopmargin=1ex, innerbottommargin=1ex, innermargin =+0cm, outermargin =+0cm, needspace={0.15\textheight}}
\newcounter{problemCounter}[chapter]
\numberwithin{problemCounter}{chapter}
\newmdtheoremenv[style=problemSty]{problem}[problemCounter]{Problema}

% Ambiente "Nota":

\mdfdefinestyle{noteSty}{backgroundcolor=white, font=\small, fontcolor=verde_UnB, linecolor=verde_UnB, linewidth=1pt, innerleftmargin=3ex, innerrightmargin=3ex, innertopmargin=2ex, innerbottommargin=2ex, innermargin =+0cm, outermargin =+0cm, needspace={0.1\textheight}}
\newcounter{noteCounter}[chapter]
\numberwithin{noteCounter}{chapter}
\newmdtheoremenv[style=noteSty]{note}[noteCounter]{Nota}

% Ambiente "Discussão" (texto complementar):

\mdfdefinestyle{discussionSty}{backgroundcolor=white, fontcolor=verde_UnB, linecolor=verde_UnB, linewidth=0pt, innerleftmargin=3ex, innerrightmargin=3ex, innertopmargin=3ex, innerbottommargin=3ex, innermargin=+0cm, outermargin =+0cm, needspace={0.25\textheight}}
\newcounter{discussionCounter}%[chapter]
%\numberwithin{discussionCounter}{chapter}
\newmdtheoremenv[style=discussionSty]{discussion}[discussionCounter]{Texto Complementar}

% Ambiente "Dedicatória" (baseado em https://tex.stackexchange.com/a/167529/91816):

\newenvironment{dedication}
  {
    \itshape             % the text is in italics
    \raggedleft          % flush to the right margin
  }
  {
    \par % end the paragraph
  }
  % space at bottom is three times that at the top
  
% Citação com autor:

\renewenvironment{quotation}
  {\small\list{}{\rightmargin=0.1cm \leftmargin=4cm \topsep=1.5\baselineskip}%
   \item\relax}
  {\endlist}

\def\signed #1{{\leavevmode\unskip\nobreak\newline
  \hbox{}\nobreak #1 %
  \hfil\parfillskip=0pt \finalhyphendemerits=0 \endgraf}}

\newsavebox\mybox
\newenvironment{citacao}[1]
  {\savebox\mybox{#1}\begin{quotation}}
  {\signed{\usebox\mybox}\end{quotation}}


\title{Modelo de livro da série \\ Ensino de Graduação da Editora UnB}
\author{VALENTE et. al, 2023} % Para mais de um autor, use: \author{autor1 \\ autor2 \\ autor3}
\date{Janeiro de 2023}

% Cabeçalhos das páginas ímpares (odd), à esquerda (LO), centro (CO) e direita (RO):

\makeatletter
\fancyhead[LO]{}
\fancyhead[CO]{\small \if@mainmatter \small \leftmark \fi}
\fancyhead[RO]{}
\makeatother

% Cabeçalhos das páginas pares (even):
% Obs.: caso haja mais de um autor, substitua \theauthor pelo nome do 1º autor, e coloque os outros em CE ou RE:

\fancyhead[LE]{\theauthor}
\fancyhead[CE]{}
\fancyhead[RE]{UNB} 

% Rodapé:

\fancyfoot[C]{\small \thepage}
%\fancyfoot[RO]{\small \nouppercase \rightmark}

% Para que \mainmatter não resete a contagem de páginas (https://tex.stackexchange.com/questions/380816/prevent-frontmatter-to-reset-counting):

\makeatletter
\renewcommand\mainmatter{%
  \cleardoublepage
  \@mainmattertrue
  \renewcommand{\thepage}{\arabic{page}}
  %\pagenumbering{arabic}% Don't reset
}
\makeatother

% Início do documento:

\begin{document}

% Ajustando espaços no sumário:

\makeatletter
\renewcommand{\@pnumwidth}{2.0em}
\renewcommand{\@tocrmarg}{2.7em}
\makeatother

% Parte Pré-Textual:

\frontmatter
%\pagestyle{empty}

% O código do texto está nos arquivos dentro da pasta "txt".
% Por exemplo, o comando %%% Física para Ciências Agrárias

%%% Capa

%%% Definindo estilo desta pagina como "vazio": 
\thispagestyle{empty}

%%% Capa: 
%%% Modelo tirado de http://tex.stackexchange.com/questions/17579/how-can-i-design-a-book-cover

\vspace*{\fill}

\begin{center}
\textbf{\color{verde_UnB}\fontsize{38pt}{45.6pt}\selectfont \textbf{PRÁTICAS EDUCATIVAS DE TRÂNSITO}}
\end{center}

\vfill

\vspace*{\fill}

\clearpage inclui o código que está no arquivo "titulo.tex" dentro da pasta txt.
% Crie seus próprios arquivos à vontade e inclua-os aqui com o comando \include{}.

%%% Física para Ciências Agrárias

%%% Capa

%%% Definindo estilo desta pagina como "vazio": 
\thispagestyle{empty}

%%% Capa: 
%%% Modelo tirado de http://tex.stackexchange.com/questions/17579/how-can-i-design-a-book-cover

\vspace*{\fill}

\begin{center}
\textbf{\color{verde_UnB}\fontsize{38pt}{45.6pt}\selectfont \textbf{PRÁTICAS EDUCATIVAS DE TRÂNSITO}}
\end{center}

\vfill

\vspace*{\fill}

\clearpage
\clearpage

\thispagestyle{empty}

\begin{center}

\begin{tabular}{r:l}
						& \textbf{\Large Universidade de Brasília}				\\
						&														\\
	{\bf Reitora} 		& Márcia Abrahão Moura									\\
	{\bf Vice-Reitor}	& Enrique Huelva										\\
						&														\\
						& \textbf{\Large Editora UnB}							\\
						&														\\
	{\bf Diretora}		& Germana Henriques Pereira								\\
						&														\\
	{\bf Conselho editorial}	&	Germana Henriques Pereira					\\
								&	Fernando César Lima Leite					\\
								&	Estevão Chaves de Rezende Martins			\\
								&	Beatriz Vargas Ramos Gonçalves de Rezende	\\
								&	Jorge Madeira Nogueira						\\
								&	Lourdes Maria Bandeira						\\
								&	Carlos José Souza de Alvarenga				\\
								&	Sérgio Antônio Andrade de Freitas			\\
								&	Verônica Moreira Amado						\\
								&	Rita de Cássia de Almeida Castro			\\
								&	Rafael Sanzio Araújo dos Anjos				\\
\end{tabular}

\end{center}

\clearpage

\clearpage

% 1ª página de cada capítulo tem estilo "vazio" (sem cabeçalho ou rodapé):
\thispagestyle{empty}

\begin{center}

\vspace*{\fill}

\textbf{\color{verde_UnB}\fontsize{30pt}{35pt}\selectfont \textbf{Práticas Educativas de Trânsito}}


\vfill

{\textbf{\fontsize{13pt}{15.6pt}\selectfont \centering  Mário Diego Rocha Valente}}

\vspace{1cm}

{\textbf{\fontsize{13pt}{15.6pt}\selectfont \centering  Marcelo Santos}}

\vspace*{\fill}

\end{center}

\newpage
\clearpage

\thispagestyle{empty}

\begin{center}

\begin{tabular}{r:l}
						& \textbf{\Large Universidade de Brasília}				\\
						&														\\
	{\bf Preparação e revisão} 		& Tiago de Aguiar Rodrigues					\\
	{\bf Diagramação}				& Wladimir de Andrade Oliveira				\\
						&														\\
						& \textcopyright 2018 Editora Universidade de Brasília	\\
						&														\\
						& Direitos exclusivos para esta edição:					\\
						&	Editora Universidade de Brasília					\\
						&	SCS, quadra 2, bloco C, nº 78, edifício OK,			\\
						&	2º andar, CEP 70302-4200							\\
						&	Site: \url{www.editora.unb.br}						\\
						&	E-mail: \href{mailto:contatoeditora@unb.br}{contatoeditora@unb.br}	\\							&																	\\
						& Todos os direitos reservados. \\
						& Nenhuma parte desta publicação	\\														& poderá ser armazenada ou reproduzida \\
						& por qualquer meio sem a autorização \\
						& por escrito da Editora. \\
						&		\\
						& {\bf Edital Livros Didáticos}	\\
						& Esta obra foi publicada com recursos provenientes	\\
						& do Edital DEG/UnB nº00/2023.
\end{tabular}

\end{center}

\newpage

\clearpage

\thispagestyle{empty}

\vspace*{\fill}

\begin{dedication}
    A minha avozinha Renilda, \\
    que me ensinou a bordar, tocar violão, \\
    cuidar de peixes de aquário, produzir licor \\
    e montar um orquidário, \\
    e de quem herdei o diletantismo.    \vspace{\baselineskip}
%    \usefont{T1}{LobsterTwo-LF}{bx}{it}
%    Leonardo
\end{dedication}

% Para inserir várias dedicatórias, descomente as duas linhas acima para colocar a assinatura do autor e use o modelo abaixo para fazer uma outra dedicatória:

%\vspace{5\baselineskip}

%\begin{dedication}
%	
%    \vspace{\baselineskip}
%    \usefont{T1}{LobsterTwo-LF}{bx}{it}
%    Autor 2
%\end{dedication}
\cleardoublepage

\chapter*{Agradecimentos}
\thispagestyle{empty}

\hfill %\hspace{0.5cm}

Ao Decanato de Ensino de Graduação da UnB, pelo processo de seleção de livros didáticos.

A toda a equipe da Editora UnB, pela oportunidade e pelo suporte, em especial ao revisor Tiago Rodrigues.

Ao Centro de Informática da UnB, pelo suporte tecnológico.

Ao professor Olavo Leopoldino da Silva Filho, que foi meu coautor do livro Física para Ciências Agrárias e Ambientais.

À equipe do Overleaf, em especial a Lian Tze Lim pela solução de vários problemas.

A incontáveis desenvolvedores de pacotes para TeX/LaTeX e participantes de fóruns pelos quais resolvemos vários problemas relacionados a esses sistemas de editoração.

Ao fórum TeX - LaTeX do site Stack Exchange, no qual até mesmo os próprios desenvolvedores dos pacotes frequentemente responderam às questões.
\cleardoublepage

% Lista de símbolos/variáveis:

\chapter*{Lista de símbolos}
\thispagestyle{empty}

\vspace{0.5cm}

\begin{longtable}{|c|l|} % "|" = "linha vertical", "c" = "centro", "l" = "esquerda", "r" = "direita"
 \hline
  \textbf{Variável} & \textbf{Grandeza representada} \\ \hline
  $A$ & área \\ \hline
  $\alpha$, $\beta$, $\gamma$ & radiações \\ \hline
  $a$ & aceleração \\ \hline
  $\mathcal{B}$ & campo magnético \\ \hline
  $D$ & diâmetro \\ \hline
  $d$ & deslocamento, distância \\ \hline
  $\textgoth{D}$ & dose absorvida \\ \hline
  $\Delta{X}$ & variação ou margem de erro da medida $X$ \\ \hline
  $\D{X}$ & elemento infinitesimal, diferencial da grandeza $X$ \\ \hline
  $\DI{X}$ & diferencial inexata da grandeza $X$ \\ \hline
  $E$ & energia \\ \hline
  $\mathcal{E}$ & campo elétrico \\ \hline
  $\epsilon$  & permissividade elétrica \\ \hline
  $\eta$ & coeficiente de viscosidade \\ \hline
  $F$ & força \\ \hline
  $\phi$   & ângulo, ângulo azimutal (coordenadas polares ou esféricas) \\ \hline
  $h$ & altura, elevação \\ \hline
  $H$ & dose equivalente \\ \hline
  $I$ & corrente elétrica \\ \hline
  $K$ & energia cinética \\ \hline
  $k$ & constante elástica \\ \hline
  $k$ & número de onda, vetor de onda ($\vec{k}$) \\ \hline
  $\kappa$ & constante eletrostática de um meio \\ \hline
  $\kappa_0$ & constante de Coulomb (constante eletrostática do vácuo) \\ \hline
  $L$ & lado \\ \hline
  $L$ & momento angular \\ \hline
  $l$ & comprimento \\ \hline
  $m$ & massa \\ \hline
  $\mu$ & permeabilidade magnética \\ \hline
  $\mu_{din}$ & coeficiente de atrito dinâmico \\ \hline
  $\mu_{est}$ & coeficiente de atrito estático \\ \hline
  $N$ & força normal \\ \hline
  $\hat{n}$ & vetor normal unitário \\ \hline
  $P$ & peso \\ \hline
  $\mathscr{P}$ & potência \\ \hline
  $p$ & momento linear \\ \hline
  $\mathscr{p}$ & pressão \\ \hline
  $Q$ & calor \\ \hline
  $\mathscr{Q}$ & vazão \\ \hline
  $\textgoth{Q}$ & fator de qualidade \\ \hline  
  $q$ & carga elétrica \\ \hline
  $r$ & coordenada radial \\ \hline
  $r$, $\theta$ & coordenadas polares \\ \hline
  $r$, $\theta$, $\phi$ & coordenadas esféricas \\ \hline  
  $R$ & raio \\ \hline
  $\mathscr{R}$ & resistência elétrica \\ \hline
  $\rho$ & densidade \\ \hline
  $T$ & temperatura \\ \hline
  $\tau$ & torque, momento de força \\ \hline
  $\omega$ & velocidade angular \\ \hline
  $\mathscr{T}$ & força de tensão \\ \hline
  $T$ & temperatura \\ \hline
  $U$ & energia potencial \\ \hline
  $\mathscr{U}$ & diferença de potencial elétrico (voltagem) \\ \hline
  $v$ & velocidade \\ \hline
  $\mathscr{W}$ & trabalho \\ \hline 
  $x$ & deformação linear \\ \hline
  $\bar{x}$ & média de x \\ \hline
  $\vec{x}$ & vetor x \\ \hline
  $\mathscr{x}$, $\mathscr{y}$, $\mathscr{z}$ & eixos de um plano cartesiano \\ \hline
  $\textgoth{X}$ & exposição \\ \hline
\end{longtable}
\cleardoublepage

% Lista de figuras:

\listoffigures \thispagestyle{empty}
\cleardoublepage

% Lista de tabelas:

\listoftables \thispagestyle{empty}
\cleardoublepage

% Sumário:

\tableofcontents\thispagestyle{empty}
\cleardoublepage 

% Parte Textual:

\mainmatter

\chapter{Segurança Viária}

\section{Aspectos Históricos}

O trânsito causa a morte de aproximadamente 1,3 milhão de pessoas e a incapacitação de milhões de outras. Cerca de 90\% das mortes e das lesões causadas pelo trânsito ocorrem em países de baixa ou média renda. Há vários anos, tanto as Nações Unidas como seus Estados Membros reconhecem que as lesões e mortes no causadas pelo trânsito são um problema. Entretanto, foi na década passada que o tema começou a ganhar o protagonismo que merece entre os assuntos mais urgentes que figuram nos programas mundiais para a saúde e o desenvolvimento internacionais.\vskip0.3cm


Na esteira da publicação do Relatório mundial sobre prevenção das lesões causadas pelo trânsito em 2004 da OMS e do Banco Mundial, várias resoluções da Assembleia Mundial da Saúde conclamaram os Estados Membros a darem prioridade à segurança no trânsito e a adotarem medidas consideradas eficazes para reduzir as mortes no trânsito. Como prova de que esse apoio político vem crescendo, a comunidade internacional tem organizado vários eventos mundiais importantes, nos quais as organizações governamentais tem se envolvido de forma intensa e visível. Esse tipo de iniciativa tem potencializado os esforços por salvar vidas nas vias de todo o mundo.\vskip0.3cm


Assim, nesse contexto, será mostrado um cronograma histórico dos esforços mundiais acerca da segurança viária global.\vskip0.3cm

 
Cronograma dos Esforços Mundiais Pela Segurança no Trânsito 

\newpage
\begin{itemize}
\item \textbf{Agosto de 2003}, o Secretário Geral das Nações Unidas publica o primeiro relatório sobre a crise mundial da segurança no trânsito;
\item \textbf{Novembro de 2003}, a Assembleia Geral das Nações Unidas, em sua 58ª Seção, adota a resolução A/RES/58/9, sobre a crise mundial da segurança no trânsito;
\item \textbf{Abril de 2004}, o Dia Mundial da Saúde é celebrado com o lema “A segurança no trânsito não é acidental”; a OMS e o Banco Mundial publicam o Relatório Mundial sobre Prevenção das Lesões Causadas pelo Trânsito;
\item \textbf{Maio de 2004}, a 57ª Assembleia Mundial da Saúde adota a resolução WHA57.10, sobre segurança no trânsito e saúde;
\item \textbf{Outubro de 2004}, criação do Grupo de colaboração das Nações Unidas para a segurança no trânsito, em Genebra, que conta com várias organizações não governamentais como membros fundadores;
\item \textbf{Outubro de 2005}, a Assembleia Geral das Nações Unidas, em sua 60ª Sessão, adota a resolução A/RES/60/5, que convida os Estados Membros a instituírem um Dia Mundial em Memória das Vítimas do Trânsito, um evento celebrado no terceiro domingo de novembro de cada ano para lembrar os mortos e feridos em acidentes de trânsito;
\item Junho de 2006, Lançamento, em Londres, da campanha Make Roads Safe;
\item \textbf{Junho de 2006}, a OMS publica o Relatório, Capacetes: um manual de segurança no trânsito para os gestores e profissionais de saúde, Conselhos Práticos aos profissionais de segurança viária rodoviário;
\item \textbf{Abril de 2007}, Primeira Semana Mundial sobre Segurança no Trânsito das Nações Unidas; a OMS publica Youth and Road Safety;
\item \textbf{Maio de 2007}, a Associação Para Viagens Rodoviárias Internacional e a OMS publicaram o Relatório intitulado, Os Rostos por trás dos Números: testemunhos das vítimas de acidentes de trânsito e de seus familiares, no qual são apresentados 22 relatos narrados por membros de uma família de vítimas de acidentes de trânsito;
\item \textbf{Junho de 2007}, a OMS publica o Relatório, Beber e Dirigir: manual de segurança viária para profissionais de trânsito e de saúde, propondo boas práticas, soluções simples e de baixo custo para evitar a associação bebida versus direção veicular;
\item \textbf{Abril de 2008}, a OMS publica o Relatório, Gestão da velocidade: um manual de segurança viária para gestores e profissionais da área, propondo soluções simples, eficazes, e de baixo custo para velocidades excessivas;
\item \textbf{Maio de 2009}, Primeira reunião mundial de organizações não governamentais em defesa da segurança no trânsito e das vítimas de acidentes, sediada em Bruxelas, e as organizações não governamentais adotam a Declaração de Bruxelas;
\item \textbf{Junho de 2009}, a OMS publica o Relatório sobre a Situação Mundial da Segurança no Trânsito;
\item \textbf{Novembro de 2009}, Primeira Conferência Ministerial Mundial sobre Segurança no Trânsito, ocorrida em Moscou na Rússia, onde os Estados Membros adotam a Declaração de Moscou;
\item \textbf{Março de 2010}, a Assembleia Geral das Nações Unidas, em sua 64ª seção, adota a resolução A/RES/64/255, em virtude da qual se proclama oficialmente a Década de Ação pela Segurança no Trânsito 2011-2020;
\item \textbf{Março de 2011}, Segunda reunião mundial de organizações não governamentais em defesa da segurança no trânsito e às vítimas de acidentes, sediada em Washington, D.C;
\item \textbf{Maio de 2011}, Lançamento da Década de Ação pela Segurança no Trânsito 2011-2020, celebrado no mundo inteiro;
\item \textbf{Maio de 2013}, a OMS publica o Relatório, Segurança de Pedestres: Manual de segurança viária para gestores e profissionais da área, cujo objetivo é fornecer informações praticas sobre como planejar, implementar e avaliar um programa de segurança de pedestres;
\item \textbf{Abril de 2015}, a OMS publica o Relatório, Cobertura de Segurança no Trânsito: um guia pra jornalistas, cujo objetivo é refletir as experiências e lições apreendidas dos workshops com jornalistas e editores e especialistas em saúde pública para aprimorar os relatórios sobre segurança nas estradas;
\item \textbf{Maio de 2015}, a OMS publica o Relatório, Fortalecendo a legislação de Segurança Viária: um guia para realização de workshops sobre legislação de segurança viária, cujo objetivo de auxiliar os Gestores estaduais e Municipais.
\item \textbf{Novembro de 2015}, Segunda Conferência Global de Alto Nível sobre Segurança no Trânsito, ocorrida em Brasília no Brasil, onde os Estados Membros adotam a Declaração de Estocolmo;
\item \textbf{Setembro de 2015}, 193 Estados Membros da ONU reuniram-se na sede da instituição, em Nova York, e decidiram um plano de ação para erradicar a pobreza, proteger o planeta e garantir que as pessoas alcancem a paz e a prosperidade: a Agenda 2030 para o Desenvolvimento Sustentável, a qual contém o conjunto de 17 Objetivos de Desenvolvimento Sustentável (ODS) e 169 metas integradas e transformadoras;
\item \textbf{Abril de 2016}, OMS publica o Relatório, Uso de Drogas e Segurança no Trânsito: um resumo de políticas, cujo objetivo e fornecer informações atualizadas sobre uso de drogas para respaldar decisões fundamentadas sobre segurança no trânsito e políticas contra as drogas, sugerindo intervenções para redução dos acidentes, lesões e morte relacionadas ao uso de drogas.
\item \textbf{Fevereiro de 2020}, Terceira Conferência Ministerial Mundial sobre Segurança no Trânsito, ocorrida em Estocolmo na Suécia, onde os Estados Membros adotam a Declaração de Estocolmo;
\end{itemize}





Respeitar os limites de velocidade, não associar o consumo de bebidas alcoólicas à direção de veículos, não se distrair com o celular ao volante, usar capacete, cinto de segurança e cadeirinha para crianças. De acordo com a Organização Mundial de Saúde (WHO), esses são os principais fatores para a segurança viária, todos amplamente conhecidos e também desobedecidos no mundo todo.\vskip0.3cm

Quanto maior a velocidade média do trânsito, maior é a probabilidade de uma colisão e maior a gravidade das suas consequências – especialmente para os pedestres, ciclistas e motociclistas. Os países que foram bem sucedidos na redução das mortes por lesões no trânsito alcançaram tal objetivo por terem dado prioridade à segurança, ao legislar sobre velocidade e sua redução.\vskip0.3cm

A violência no trânsito virou uma epidemia. Não figurava entre as principais causas de óbito em 2010, mas assumiu a 10ª colocação em 2015. A previsão, caso mudanças não sejam rapidamente implementadas, é um salto para o 7º lugar até 2030.\vskip0.3cm

Aproximadamente 3,4 mil pessoas morrem diariamente em acidentes de trânsito no mundo. As principais causas de óbitos no mundo por acidentes de trânsito.\vskip0.3cm

Os dados globais referentes à segurança viária assustam: foram quase 50 milhões de feridos e mais de 1,25 milhão de vítimas em acidentes em vias do mundo todo em 2013. A violência no trânsito atinge, principalmente, uma geração jovem e economicamente ativa (entre 15 e 29 anos). Cerca de dois terços das vítimas são homens. Além do impacto nas famílias e comunidades, a OMS calcula que as perdas econômicas abocanhem até 5\% da renda dos países em desenvolvimento. Para a entidade, é impossível excluir do cálculo custos indiretos, como perda de produtividade, danos a veículos e propriedades, qualidade reduzida de vida e outros fatores que pesam para a sociedade.\vskip0.3cm

Ainda que seja um problema mundial, a vulnerabilidade é maior dependendo de onde você está e de quem você é. Os dados da OMS mostram que quase 50\% das vítimas são motoqueiros, pedestres e ciclistas. A taxa de mortalidade nos países de baixa e média renda, que detém pouco mais da metade da frota de todo o mundo, é de 90\%.\vskip0.3cm

De acordo com a OMS e a Organização Pan-Americana da Saúde (OPAS), a questão da segurança viária também é alarmante nas Américas, cuja taxa de óbitos por 100 mil habitantes é de 15,9. Em relatório específico, as organizações recomendam à região, caracterizada pela forte desigualdade de renda e educação, o investimento em infraestrutura, a reforma da legislação sobre segurança e uma fiscalização eficaz para melhorar o comportamento dos usuários das vias públicas e reduzir as baixas no trânsito.\vskip0.3cm


No Brasil, o Relatório da Segurança Viária no Brasil 2019,  traz informações atualizadas, obtidas através do cruzamento de dados da Associação Nacional dos Transportes Públicos (ANTP), da Confederação do Transporte (CNT), do Departamento de Informática do Sistema Único de Saúde (Datasus), do Depatamento Nacional de Trânsito (Denatran), do Departamento Nacional de Infraestrutura de Transportes (DNIT), do Instituto Brasileiro de Geografia e Estatística (IBGE), do Instituto de Pesquisa Econômica Aplicada (IPEA) e da Organização Mundial de Saúde (OMS).\vskip0.3cm

Houve uma redução de 16\% no número de mortes entre 2010 e 2015. O índice de óbitos por 100 habitantes, um indicador que reflete a magnitude do problema, atingiu 19,2, o melhor resultado desde 2004. Ainda em 2015, os acidentes fatais concentraram-se em 13 estados brasileiros, que representam 80\% do total. São Paulo lidera o ranking do estados em número absolutos de óbitos, mas obtém o 4º menor indicador por 100 mil habitantes do país, atrás apenas de Amazonas, Amapá e Rio de Janeiro. Segundo a Associação Brasileira de Medicina de Tráfego (Abramet), os acidentes de trânsito já são a segunda causa de morte não natural evitável na população.\vskip0.3cm


\cleardoublepage

\chapter{Educação de Trânsito}
\section{Conceitos Fundamentais}


\section{Educação de Trânsito via CFB}

A educação para o trânsito está assegurada na Constituição Federal Brasileira(CFB) de 1988, no Art. 23 e inciso XII e assim estão definidas as competências dos entes federados, tendo em vista a dimensão do pacto federativo: “[...] é competência comum da União, dos Estados, do Distrito Federal e dos Municípios: estabelecer e implantar política de educação para a segurança do trânsito”.  Sendo, ainda ratificada esta responsabilidade, no parágrafo único do artigo, quando dispõe que leis complementares fixarão as normas para a cooperação entre os entes federativos. Então, observa-se que a proposição da educação e segurança no trânsito é compreendida enquanto política pública a ser efetivada pelos diferentes entes federados.\vskip0.3cm


Incorporar a educação para o trânsito no Ensino Fundamental supõe produzir conhecimentos científicos que atendam as demandas desses sujeitos, sobretudo referentes aos direitos garantidos na legislação e ainda não efetivados. Outro aspecto previsto na Constituição Federal de 1988, é que no Art. 210 estabelece que: “Serão fixados conteúdos mínimos para o ensino fundamental, de maneira a assegurar formação básica comum e respeito aos valores culturais e artísticos, nacionais e regionais”.\vskip0.3cm

Há que se considerar, ainda, o artigo nº 205 da Constituição Federal (BRASIL, 1988), ao propor “[...] o pleno desenvolvimento da pessoa humana [...]”. Isso pressupõe que a educação para o trânsito pode servir de instrumento poderoso de desenvolvimento da pessoa humana na busca do exercício da cidadania e preservação à vida.\vskip0.3cm

A Constituição do Estado do Pará, promulgada em 05 de outubro de 1989, inspirada nos princípios constitucionais da República Federativa do Brasil, no seu Capítulo II, que trata da Competência do Estado, já previa a necessidade da implantação da educação de trânsito, conforme texto constitucional onde no seu mencina no Art. 17 - É competência comum do Estado e dos Municípios, com a União: XII - estabelecer e implantar política de educação para a segurança do trânsito. \vskip0.3cm

Assim, a Educação de Trânsito, prevista na Constituição Estadual, ocupa o seu lugar na Lei maior que representa a vontade do povo, permitindo atribuir à mesma um grau de importância tamanha, devido esta modalidade de educação aparecer como uma responsabilidade ao lado da saúde, cultura, educação, ciência, meio ambiente, entre outros, permitindo compreender a importância da mesma, principalmente, por tratar da segurança no trânsito da pessoas que vivem e transitam pelo vasto território paraense. 


\section{Educação de Trânsito via CTB}

Na década de 1910, começa a elaboração das legislações, com a finalidade de disciplinar o trânsito e estimular a criação de estradas para ligar as regiões brasileiras através do uso de veículos automotores como meio de transporte. Os códigos de trânsito brasileiro existentes, além de buscarem disciplinar o trânsito, atuavam nas questões atinentes à atividade policial, sinalização e segurança no trânsito, produzindo os seus efeitos até 1996, quando, em substituição ao Código Nacional de Trânsito, foi promulgado o Código de Trânsito Brasileiro (CTB) atualmente em vigor.\vskip0.3cm

O novo CTB foi criado pela Lei nº 9.503 e sancionada pela Presidência da República em 23 do mês de setembro do ano de 1997. E está definido no seu Art. 1º, § 2º, que o “[...] trânsito, em condições seguras, é um direito de todos e dever dos órgãos e entidades componentes do Sistema Nacional de Trânsito, a estes cabendo, no âmbito das respectivas competências, adotar as medidas destinadas a assegurar esse direito” CTB (BRASIL, 1997, p.1).\vskip0.3cm

Pelo exposto, sabe-se que a lei trouxe muitas inovações para a regulamentação do trânsito no Brasil, instituída por leis e decretos, além de resoluções dos conselhos competentes. Assim, o CTB tem a finalidade de regular o sistema nacional de transito, tendo em vista as normas gerais de circulação e conduta para motoristas e pedestres, além de estabelecer a obrigatoriedade de educação para o trânsito.\vskip0.3cm


O CTB é um documento extenso, constituído inicialmente com 341 artigos e seus respectivos parágrafos, organizados em 20 capítulos que definem atribuições das diversas autoridades e órgãos ligados ao trânsito do Brasil. 



\section{Educação de Trânsito via BNCC}
\clearpage
%\fancyfoot[CO,CE]{}
\fancyfoot[LO]{\small}
\fancyfoot[RO]{\small}
%\fancyfoot[CE]{\Author}
\fancyfoot[LE]{\small}
\fancyfoot[RE]{\small}


\cleardoublepage

\chapter{Práticas Educativas de Trânsito}
\section{Transitando no Bares}


O Projeto Transitando nos Bares teve início em junho de 2008, em virtude da entrada em vigor da Lei nº 11.705 (\textbf{Lei da Seca}).\vskip0.3cm

Uma equipe de Técnicos e Agentes de Educação da Coordenadoria de Educação do Detran/PA, realizou atividades com abordagens educativas, interagindo com o público que frequenta alguns bares do Município de Belém, capital do Estado do Pará, com o objetivo de sensibilizar essa população sobre o perigo existente na associação de álcool e direção. \vskip0.3cm

A avaliação dessas ações mostrou resultados positivos na medida em que as pessoas abordadas demonstraram ser favoráveis à ideia da campanha, além disso, os proprietários dos estabelecimentos visitados demonstraram o interesse de firmar parceria com o DETRAN/PA para trabalhar na prevenção da acidentalidade e mortalidade no trânsito, pois compreenderam a necessidade de campanhas que estimulem o cumprimento da lei associado à Educação para o Trânsito que é um fator indispensável na preservação da vida.\vskip0.3cm

Em 2009, as atividades continuaram sendo realizadas em alguns bares em Belém, sendo trabalhadas em quatro meses deste ano, com boa receptividade pelo público. O projeto foi interrompido por dois anos, entretanto, com a revisão da Lei 12.760 de dezembro de 2012 e a aplicação da resolução nº 432, de 23 de Janeiro de 2013, que tornaram mais rígidas as penalidades sobre o condutor flagrado sob o efeito de bebida alcoólica, as discussões sobre a reforma desse trabalho de conscientização foram retomadas.\vskip0.3cm

Em julho de 2015, no município de \textbf{Salinópolis} foi feito o Projeto Piloto, onde ocorreu uma intensa campanha de conscientização de veranistas, que com o apoio de donos de bares e restaurantes, aliado a outras estratégias contribuiu para um veraneio sem acidentes com vítimas fatais. A partir dos resultados iniciais de Salinópolis, o Projeto foi executado em Belém e nos Municípios de Bragança, Tucuruí, Marabá, Parauapebas, Santarém, Abaetetuba, Altamira e Itaituba.\vskip0.3cm


Em Belém o projeto foi realizado na Área 1: Av. V. Souza Franco (Entorno da “Doca”, onde foram atendidos os bares Meu Garoto, Du Pará e Doca Grill, Dom Bar, Devassa e Old School. No ano de 2017 o projeto será expandido para as demais áreas e para outros Municípios Pólos no Interior do Estado.


\subsection{Metodologia}







\newpage
\section{Abordagem Educativa em RodoVias}
\subsection{Planejamento da Abordagem}

\begin{itemize}
\item Antes de cada ação o coordenador deverá reunir-se com a equipe e parceiros para detalhar a forma de  trabalho;
\item Deverá ser atribuido as tarefas e dividir as funções para cada membro da equipe;
\item Identificar os parceiros e a atuação de cada um no momento da abordagem;
\item Comunicar a linguagem que será utilizada durante a ação, de modo que permita as pessoas identificarem a campanha;
\item Orientar sobre o preenchimento de planilhas e outros materiais;
\item Definir os locais de atuação, levantar a necessidade de materiais e montar a logística de operação;
\item Definir o horário de trabalho com os parceiros da ação. As equipes devem chegar meia hora antes e cumprir o horário até o final.  
\end{itemize}

\subsection{Apresentação Pessoal}

\begin{itemize}
\item É obrigatório o uso de uniforme como forma de padronizar o trabalho que será realizado;
\item Aos homens é importante apresentar-se diariamente com a barba feita;
\item Às mulheres, recomenda-se o uso moderado de maquiagem e jóias/bijuterias, que deverá ser discreta; e não esquecer o cuidado em prender cabelos longos;
\item É recomendado um cuidado redobrado com a higiene pessoal;
\item É indispensável o cuidado para não fumar, nem beber, antes da realização da ação, em virtude de fazermos parte de um programa de educação que veicula comportamento seguro e qualidade de vida;
\item Durante os trabalhos na via evitar o uso de aparelhos celular ou ficar conversando em grupos na margem da via;
\item Evitar tirar fotos com celular de condutores ou passageiros, pois será designado uma pessoa da equipe para tal função.
\end{itemize}








\subsection{Abordagem Educativa de Trânsito}

\begin{itemize}
\item A  maneira de abordar é fundamental para dar credibilidade a ação, devendo ser objetivo na mensagem a que se pretenda repassar;
\item Sempre deveremos utilizar crachá, chamar as pessoas de Sr. ou Srª, cumprimentar com bom dia, boa tarde ou boa noite, identificar-se como integrante  da equipe de educação de trânsito;
\item A abordagem não pode ter caráter crítico ou repreensivo, deverá “conquistar”, “sensibilizar” e “promover”  a idéia de um comportamento seguro e valorização da vida;
\item Importante lembrar que estaremos abordando pessoas que se encontram a passeio, e que por isso poderão mostrar-se resistentes a orientações, assim deve se treinar as abordagens;
\item Para as abordagens cada integrante da equipe deverá conhecer o material a ser utilizado e tirar as dúvidas antes da ação;
\item Os membros da equipe de educação deverão posicionar-se às margens da via, sempre do lado do carona;
\item Em hipótese alguma deve-se “encostar” no veículo a ser abordado ou colocar a mão como forma de apoio;
\item A ordem de parada e o balizamento dos veículos deverá ser efetuada pelo Agente de Trânsito, que dará instruções à equipe visando a segurança na via; 
\item Em cada abordagem deve-se observar as condutas inadequadas e orientar para a correção no momento da parada;
\item Para as condutas corretas elogiar o condutor e passageiros; 
\item Toda orientação deve ser cuidadosa como forma de estimular o bom comportamento no trânsito, com o cumprimento das normas de circulação e a observância dos valores essenciais para uma boa educação no trânsito;
\item É imprescindível a comunicação entre a equipe durante as abordagens. Todos devem observar o andamento da ação e tudo que vier a ocorrer, que fuja do planejado, deverá ser comunicado imediatamente ao coordenador;
\end{itemize}





\cleardoublepage

\chapter{Abordagem Educativa em Rodovias}
\subsection{Planejamento da Abordagem}

\begin{itemize}
\item Antes de cada ação o coordenador deverá reunir-se com a equipe e parceiros para detalhar a forma de  trabalho;
\item Deverá ser atribuido as tarefas e dividir as funções para cada membro da equipe;
\item Identificar os parceiros e a atuação de cada um no momento da abordagem;
\item Comunicar a linguagem que será utilizada durante a ação, de modo que permita as pessoas identificarem a campanha;
\item Orientar sobre o preenchimento de planilhas e outros materiais;
\item Definir os locais de atuação, levantar a necessidade de materiais e montar a logística de operação;
\item Definir o horário de trabalho com os parceiros da ação. As equipes devem chegar meia hora antes e cumprir o horário até o final.  
\end{itemize}

\subsection{Apresentação Pessoal}

\begin{itemize}
\item É obrigatório o uso de uniforme como forma de padronizar o trabalho que será realizado;
\item Aos homens é importante apresentar-se diariamente com a barba feita;
\item Às mulheres, recomenda-se o uso moderado de maquiagem e jóias/bijuterias, que deverá ser discreta; e não esquecer o cuidado em prender cabelos longos;
\item É recomendado um cuidado redobrado com a higiene pessoal;
\item É indispensável o cuidado para não fumar, nem beber, antes da realização da ação, em virtude de fazermos parte de um programa de educação que veicula comportamento seguro e qualidade de vida;
\item Durante os trabalhos na via evitar o uso de aparelhos celular ou ficar conversando em grupos na margem da via;
\item Evitar tirar fotos com celular de condutores ou passageiros, pois será designado uma pessoa da equipe para tal função.
\end{itemize}



\subsection{Abordagem Educativa de Trânsito}

\begin{itemize}
\item A  maneira de abordar é fundamental para dar credibilidade a ação, devendo ser objetivo na mensagem a que se pretenda repassar;
\item Sempre deveremos utilizar crachá, chamar as pessoas de Sr. ou Srª, cumprimentar com bom dia, boa tarde ou boa noite, identificar-se como integrante  da equipe de educação de trânsito;
\item A abordagem não pode ter caráter crítico ou repreensivo, deverá “conquistar”, “sensibilizar” e “promover”  a idéia de um comportamento seguro e valorização da vida;
\item Importante lembrar que estaremos abordando pessoas que se encontram a passeio, e que por isso poderão mostrar-se resistentes a orientações, assim deve se treinar as abordagens;
\item Para as abordagens cada integrante da equipe deverá conhecer o material a ser utilizado e tirar as dúvidas antes da ação;
\item Os membros da equipe de educação deverão posicionar-se às margens da via, sempre do lado do carona;
\item Em hipótese alguma deve-se “encostar” no veículo a ser abordado ou colocar a mão como forma de apoio;
\item A ordem de parada e o balizamento dos veículos deverá ser efetuada pelo Agente de Trânsito, que dará instruções à equipe visando a segurança na via; 
\item Em cada abordagem deve-se observar as condutas inadequadas e orientar para a correção no momento da parada;
\item Para as condutas corretas elogiar o condutor e passageiros; 
\item Toda orientação deve ser cuidadosa como forma de estimular o bom comportamento no trânsito, com o cumprimento das normas de circulação e a observância dos valores essenciais para uma boa educação no trânsito;
\item É imprescindível a comunicação entre a equipe durante as abordagens. Todos devem observar o andamento da ação e tudo que vier a ocorrer, que fuja do planejado, deverá ser comunicado imediatamente ao coordenador;
\end{itemize}





\cleardoublepage

% Apêndices:

\renewcommand\appendixname{APÊNDICE}
\renewcommand\appendixpagename{APÊNDICES}
\renewcommand{\appendixtocname}{APÊNDICES}

\begin{appendices}

% Formatando como em "APÊNDICE A -- Título do apêndice":
% Ref.: https://tex.stackexchange.com/a/479364/91816

\makeatletter
\def\@chapter[#1]#2{\ifnum \c@secnumdepth >\m@ne
    \refstepcounter{chapter}%
    \typeout{\thechapter.}%
    \addcontentsline{toc}{chapter}%
    {\thechapter\space\textendash\space\ #1}% <-- modification
  \else
    \addcontentsline{toc}{chapter}{#1}%
  \fi
  \chaptermark{#1}%
  \addtocontents{lof}{\protect\addvspace{10\p@}}%
  \addtocontents{lot}{\protect\addvspace{10\p@}}%
  \if@twocolumn
    \@topnewpage[\@makechapterhead{#2}]%
  \else
    \@makechapterhead{#2}%
    \@afterheading
  \fi}
\makeatother

% Reiniciando contador de capítulo e seção:

\setcounter{chapter}{0}
\setcounter{section}{0}

\chapter{Citações}

\thispagestyle{empty} 



\cleardoublepage

\chapter{Tabelas}

\thispagestyle{empty} 

A tabela \ref{tab:SI-basicas} mostra algumas unidades básicas do SI.


\cleardoublepage

\end{appendices}

% Anexos:

\renewcommand\appendixname{ANEXO}
\renewcommand\appendixpagename{ANEXOS}
\renewcommand{\appendixtocname}{ANEXOS}

\begin{appendices}

% Formatando como em "APÊNDICE A -- Título do apêndice":
% Ref.: https://tex.stackexchange.com/a/479364/91816

\makeatletter
\def\@chapter[#1]#2{\ifnum \c@secnumdepth >\m@ne
    \refstepcounter{chapter}%
    \typeout{\thechapter.}%
    \addcontentsline{toc}{chapter}%
    {\thechapter\space\textendash\space\ #1}% <-- modification
  \else
    \addcontentsline{toc}{chapter}{#1}%
  \fi
  \chaptermark{#1}%
  \addtocontents{lof}{\protect\addvspace{10\p@}}%
  \addtocontents{lot}{\protect\addvspace{10\p@}}%
  \if@twocolumn
    \@topnewpage[\@makechapterhead{#2}]%
  \else
    \@makechapterhead{#2}%
    \@afterheading
  \fi}
\makeatother

% Reiniciando contador de capítulo e seção:

\setcounter{chapter}{0}
\setcounter{section}{0}

\chapter{Figuras de exemplo}

\thispagestyle{empty} 


\cleardoublepage

\chapter{Logo LaTeX}

\thispagestyle{empty} 


\cleardoublepage

\end{appendices}

% Parte Pós-Textual:
\backmatter

% Referências (Bibliografia):

\singlespacing

\bibliographystyle{abntex2-alf}
\bibliography{main}
\cleardoublepage

% Índice Remissivo, controlado pelo comando \index{...} em meio ao texto:

\phantomsection
\printindex

\end{document}
