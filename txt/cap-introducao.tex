\chapter{Segurança Viária}

\section{Aspectos Históricos}

O trânsito causa a morte de aproximadamente 1,3 milhão de pessoas e a incapacitação de milhões de outras. Cerca de 90\% das mortes e das lesões causadas pelo trânsito ocorrem em países de baixa ou média renda. Há vários anos, tanto as Nações Unidas como seus Estados Membros reconhecem que as lesões e mortes no causadas pelo trânsito são um problema. Entretanto, foi na década passada que o tema começou a ganhar o protagonismo que merece entre os assuntos mais urgentes que figuram nos programas mundiais para a saúde e o desenvolvimento internacionais.\vskip0.3cm


Na esteira da publicação do Relatório mundial sobre prevenção das lesões causadas pelo trânsito em 2004 da OMS e do Banco Mundial, várias resoluções da Assembleia Mundial da Saúde conclamaram os Estados Membros a darem prioridade à segurança no trânsito e a adotarem medidas consideradas eficazes para reduzir as mortes no trânsito. Como prova de que esse apoio político vem crescendo, a comunidade internacional tem organizado vários eventos mundiais importantes, nos quais as organizações governamentais tem se envolvido de forma intensa e visível. Esse tipo de iniciativa tem potencializado os esforços por salvar vidas nas vias de todo o mundo.\vskip0.3cm


Assim, nesse contexto, será mostrado um cronograma histórico dos esforços mundiais acerca da segurança viária global.\vskip0.3cm

 
Cronograma dos Esforços Mundiais Pela Segurança no Trânsito 


\begin{itemize}
\item \textbf{Agosto de 2003}, o Secretário Geral das Nações Unidas publica o primeiro relatório sobre a crise mundial da segurança no trânsito;
\item \textbf{Novembro de 2003}, a Assembleia Geral das Nações Unidas, em sua 58ª Seção, adota a resolução A/RES/58/9, sobre a crise mundial da segurança no trânsito;
\item \textbf{Abril de 2004}, o Dia Mundial da Saúde é celebrado com o lema “A segurança no trânsito não é acidental”; a OMS e o Banco Mundial publicam o Relatório Mundial sobre Prevenção das Lesões Causadas pelo Trânsito;
\item \textbf{Maio de 2004}, a 57ª Assembleia Mundial da Saúde adota a resolução WHA57.10, sobre segurança no trânsito e saúde;
\item \textbf{Outubro de 2004}, criação do Grupo de colaboração das Nações Unidas para a segurança no trânsito, em Genebra, que conta com várias organizações não governamentais como membros fundadores;
\item \textbf{Outubro de 2005}, a Assembleia Geral das Nações Unidas, em sua 60ª Sessão, adota a resolução A/RES/60/5, que convida os Estados Membros a instituírem um Dia Mundial em Memória das Vítimas do Trânsito, um evento celebrado no terceiro domingo de novembro de cada ano para lembrar os mortos e feridos em acidentes de trânsito;
\item Junho de 2006, Lançamento, em Londres, da campanha Make Roads Safe;
\item \textbf{Junho de 2006}, a OMS publica o Relatório, Capacetes: um manual de segurança no trânsito para os gestores e profissionais de saúde, Conselhos Práticos aos profissionais de segurança viária rodoviário;
\item \textbf{Abril de 2007}, Primeira Semana Mundial sobre Segurança no Trânsito das Nações Unidas; a OMS publica Youth and Road Safety;
\item \textbf{Maio de 2007}, a Associação Para Viagens Rodoviárias Internacional e a OMS publicaram o Relatório intitulado, Os Rostos por trás dos Números: testemunhos das vítimas de acidentes de trânsito e de seus familiares, no qual são apresentados 22 relatos narrados por membros de uma família de vítimas de acidentes de trânsito;
\item \textbf{Junho de 2007}, a OMS publica o Relatório, Beber e Dirigir: manual de segurança viária para profissionais de trânsito e de saúde, propondo boas práticas, soluções simples e de baixo custo para evitar a associação bebida versus direção veicular;
\item \textbf{Abril de 2008}, a OMS publica o Relatório, Gestão da velocidade: um manual de segurança viária para gestores e profissionais da área, propondo soluções simples, eficazes, e de baixo custo para velocidades excessivas;
\item \textbf{Maio de 2009}, Primeira reunião mundial de organizações não governamentais em defesa da segurança no trânsito e das vítimas de acidentes, sediada em Bruxelas, e as organizações não governamentais adotam a Declaração de Bruxelas;
\item \textbf{Junho de 2009}, a OMS publica o Relatório sobre a Situação Mundial da Segurança no Trânsito;
\item \textbf{Novembro de 2009}, Primeira Conferência Ministerial Mundial sobre Segurança no Trânsito, ocorrida em Moscou na Rússia, onde os Estados Membros adotam a Declaração de Moscou;
\item \textbf{Março de 2010}, a Assembleia Geral das Nações Unidas, em sua 64ª seção, adota a resolução A/RES/64/255, em virtude da qual se proclama oficialmente a Década de Ação pela Segurança no Trânsito 2011-2020;
\item \textbf{Março de 2011}, Segunda reunião mundial de organizações não governamentais em defesa da segurança no trânsito e às vítimas de acidentes, sediada em Washington, D.C;
\item \textbf{Maio de 2011}, Lançamento da Década de Ação pela Segurança no Trânsito 2011-2020, celebrado no mundo inteiro;
\item \textbf{Maio de 2013}, a OMS publica o Relatório, Segurança de Pedestres: Manual de segurança viária para gestores e profissionais da área, cujo objetivo é fornecer informações praticas sobre como planejar, implementar e avaliar um programa de segurança de pedestres;
\item \textbf{Abril de 2015}, a OMS publica o Relatório, Cobertura de Segurança no Trânsito: um guia pra jornalistas, cujo objetivo é refletir as experiências e lições apreendidas dos workshops com jornalistas e editores e especialistas em saúde pública para aprimorar os relatórios sobre segurança nas estradas;
\item \textbf{Maio de 2015}, a OMS publica o Relatório, Fortalecendo a legislação de Segurança Viária: um guia para realização de workshops sobre legislação de segurança viária, cujo objetivo de auxiliar os Gestores estaduais e Municipais.
\item \textbf{Novembro de 2015}, Segunda Conferência Global de Alto Nível sobre Segurança no Trânsito, ocorrida em Brasília no Brasil, onde os Estados Membros adotam a Declaração de Estocolmo;
\item \textbf{Setembro de 2015}, 193 Estados Membros da ONU reuniram-se na sede da instituição, em Nova York, e decidiram um plano de ação para erradicar a pobreza, proteger o planeta e garantir que as pessoas alcancem a paz e a prosperidade: a Agenda 2030 para o Desenvolvimento Sustentável, a qual contém o conjunto de 17 Objetivos de Desenvolvimento Sustentável (ODS) e 169 metas integradas e transformadoras;
\item \textbf{Abril de 2016}, OMS publica o Relatório, Uso de Drogas e Segurança no Trânsito: um resumo de políticas, cujo objetivo e fornecer informações atualizadas sobre uso de drogas para respaldar decisões fundamentadas sobre segurança no trânsito e políticas contra as drogas, sugerindo intervenções para redução dos acidentes, lesões e morte relacionadas ao uso de drogas.
\item \textbf{Fevereiro de 2020}, Terceira Conferência Ministerial Mundial sobre Segurança no Trânsito, ocorrida em Estocolmo na Suécia, onde os Estados Membros adotam a Declaração de Estocolmo;
\end{itemize}





Respeitar os limites de velocidade, não associar o consumo de bebidas alcoólicas à direção de veículos, não se distrair com o celular ao volante, usar capacete, cinto de segurança e cadeirinha para crianças. De acordo com a Organização Mundial de Saúde (WHO), esses são os principais fatores para a segurança viária, todos amplamente conhecidos e também desobedecidos no mundo todo.\vskip0.3cm

Quanto maior a velocidade média do trânsito, maior é a probabilidade de uma colisão e maior a gravidade das suas consequências – especialmente para os pedestres, ciclistas e motociclistas. Os países que foram bem sucedidos na redução das mortes por lesões no trânsito alcançaram tal objetivo por terem dado prioridade à segurança, ao legislar sobre velocidade e sua redução.\vskip0.3cm

A violência no trânsito virou uma epidemia. Não figurava entre as principais causas de óbito em 2010, mas assumiu a 10ª colocação em 2015. A previsão, caso mudanças não sejam rapidamente implementadas, é um salto para o 7º lugar até 2030.\vskip0.3cm

Aproximadamente 3,4 mil pessoas morrem diariamente em acidentes de trânsito no mundo. As principais causas de óbitos no mundo por acidentes de trânsito.\vskip0.3cm

Os dados globais referentes à segurança viária assustam: foram quase 50 milhões de feridos e mais de 1,25 milhão de vítimas em acidentes em vias do mundo todo em 2013. A violência no trânsito atinge, principalmente, uma geração jovem e economicamente ativa (entre 15 e 29 anos). Cerca de dois terços das vítimas são homens. Além do impacto nas famílias e comunidades, a OMS calcula que as perdas econômicas abocanhem até 5\% da renda dos países em desenvolvimento. Para a entidade, é impossível excluir do cálculo custos indiretos, como perda de produtividade, danos a veículos e propriedades, qualidade reduzida de vida e outros fatores que pesam para a sociedade.\vskip0.3cm

Ainda que seja um problema mundial, a vulnerabilidade é maior dependendo de onde você está e de quem você é. Os dados da OMS mostram que quase 50\% das vítimas são motoqueiros, pedestres e ciclistas. A taxa de mortalidade nos países de baixa e média renda, que detém pouco mais da metade da frota de todo o mundo, é de 90\%.\vskip0.3cm

De acordo com a OMS e a Organização Pan-Americana da Saúde (OPAS), a questão da segurança viária também é alarmante nas Américas, cuja taxa de óbitos por 100 mil habitantes é de 15,9. Em relatório específico, as organizações recomendam à região, caracterizada pela forte desigualdade de renda e educação, o investimento em infraestrutura, a reforma da legislação sobre segurança e uma fiscalização eficaz para melhorar o comportamento dos usuários das vias públicas e reduzir as baixas no trânsito.\vskip0.3cm


No Brasil, o Relatório da Segurança Viária no Brasil 2019,  traz informações atualizadas, obtidas através do cruzamento de dados da Associação Nacional dos Transportes Públicos (ANTP), da Confederação do Transporte (CNT), do Departamento de Informática do Sistema Único de Saúde (Datasus), do Depatamento Nacional de Trânsito (Denatran), do Departamento Nacional de Infraestrutura de Transportes (DNIT), do Instituto Brasileiro de Geografia e Estatística (IBGE), do Instituto de Pesquisa Econômica Aplicada (IPEA) e da Organização Mundial de Saúde (OMS).\vskip0.3cm

Houve uma redução de 16\% no número de mortes entre 2010 e 2015. O índice de óbitos por 100 habitantes, um indicador que reflete a magnitude do problema, atingiu 19,2, o melhor resultado desde 2004. Ainda em 2015, os acidentes fatais concentraram-se em 13 estados brasileiros, que representam 80\% do total. São Paulo lidera o ranking do estados em número absolutos de óbitos, mas obtém o 4º menor indicador por 100 mil habitantes do país, atrás apenas de Amazonas, Amapá e Rio de Janeiro. Segundo a Associação Brasileira de Medicina de Tráfego (Abramet), os acidentes de trânsito já são a segunda causa de morte não natural evitável na população.\vskip0.3cm

