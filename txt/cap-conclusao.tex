\chapter{Abordagem Educativa em Rodovias}
\subsection{Planejamento da Abordagem}

\begin{itemize}
\item Antes de cada ação o coordenador deverá reunir-se com a equipe e parceiros para detalhar a forma de  trabalho;
\item Deverá ser atribuido as tarefas e dividir as funções para cada membro da equipe;
\item Identificar os parceiros e a atuação de cada um no momento da abordagem;
\item Comunicar a linguagem que será utilizada durante a ação, de modo que permita as pessoas identificarem a campanha;
\item Orientar sobre o preenchimento de planilhas e outros materiais;
\item Definir os locais de atuação, levantar a necessidade de materiais e montar a logística de operação;
\item Definir o horário de trabalho com os parceiros da ação. As equipes devem chegar meia hora antes e cumprir o horário até o final.  
\end{itemize}

\subsection{Apresentação Pessoal}

\begin{itemize}
\item É obrigatório o uso de uniforme como forma de padronizar o trabalho que será realizado;
\item Aos homens é importante apresentar-se diariamente com a barba feita;
\item Às mulheres, recomenda-se o uso moderado de maquiagem e jóias/bijuterias, que deverá ser discreta; e não esquecer o cuidado em prender cabelos longos;
\item É recomendado um cuidado redobrado com a higiene pessoal;
\item É indispensável o cuidado para não fumar, nem beber, antes da realização da ação, em virtude de fazermos parte de um programa de educação que veicula comportamento seguro e qualidade de vida;
\item Durante os trabalhos na via evitar o uso de aparelhos celular ou ficar conversando em grupos na margem da via;
\item Evitar tirar fotos com celular de condutores ou passageiros, pois será designado uma pessoa da equipe para tal função.
\end{itemize}



\subsection{Abordagem Educativa de Trânsito}

\begin{itemize}
\item A  maneira de abordar é fundamental para dar credibilidade a ação, devendo ser objetivo na mensagem a que se pretenda repassar;
\item Sempre deveremos utilizar crachá, chamar as pessoas de Sr. ou Srª, cumprimentar com bom dia, boa tarde ou boa noite, identificar-se como integrante  da equipe de educação de trânsito;
\item A abordagem não pode ter caráter crítico ou repreensivo, deverá “conquistar”, “sensibilizar” e “promover”  a idéia de um comportamento seguro e valorização da vida;
\item Importante lembrar que estaremos abordando pessoas que se encontram a passeio, e que por isso poderão mostrar-se resistentes a orientações, assim deve se treinar as abordagens;
\item Para as abordagens cada integrante da equipe deverá conhecer o material a ser utilizado e tirar as dúvidas antes da ação;
\item Os membros da equipe de educação deverão posicionar-se às margens da via, sempre do lado do carona;
\item Em hipótese alguma deve-se “encostar” no veículo a ser abordado ou colocar a mão como forma de apoio;
\item A ordem de parada e o balizamento dos veículos deverá ser efetuada pelo Agente de Trânsito, que dará instruções à equipe visando a segurança na via; 
\item Em cada abordagem deve-se observar as condutas inadequadas e orientar para a correção no momento da parada;
\item Para as condutas corretas elogiar o condutor e passageiros; 
\item Toda orientação deve ser cuidadosa como forma de estimular o bom comportamento no trânsito, com o cumprimento das normas de circulação e a observância dos valores essenciais para uma boa educação no trânsito;
\item É imprescindível a comunicação entre a equipe durante as abordagens. Todos devem observar o andamento da ação e tudo que vier a ocorrer, que fuja do planejado, deverá ser comunicado imediatamente ao coordenador;
\end{itemize}




