\chapter{Educação de Trânsito}
\section{Conceitos Fundamentais}


\section{Educação de Trânsito via CFB}

A educação para o trânsito está assegurada na Constituição Federal Brasileira(CFB) de 1988, no Art. 23 e inciso XII e assim estão definidas as competências dos entes federados, tendo em vista a dimensão do pacto federativo: “[...] é competência comum da União, dos Estados, do Distrito Federal e dos Municípios: estabelecer e implantar política de educação para a segurança do trânsito”.  Sendo, ainda ratificada esta responsabilidade, no parágrafo único do artigo, quando dispõe que leis complementares fixarão as normas para a cooperação entre os entes federativos. Então, observa-se que a proposição da educação e segurança no trânsito é compreendida enquanto política pública a ser efetivada pelos diferentes entes federados.\vskip0.3cm


Incorporar a educação para o trânsito no Ensino Fundamental supõe produzir conhecimentos científicos que atendam as demandas desses sujeitos, sobretudo referentes aos direitos garantidos na legislação e ainda não efetivados. Outro aspecto previsto na Constituição Federal de 1988, é que no Art. 210 estabelece que: “Serão fixados conteúdos mínimos para o ensino fundamental, de maneira a assegurar formação básica comum e respeito aos valores culturais e artísticos, nacionais e regionais”.\vskip0.3cm

Há que se considerar, ainda, o artigo nº 205 da Constituição Federal (BRASIL, 1988), ao propor “[...] o pleno desenvolvimento da pessoa humana [...]”. Isso pressupõe que a educação para o trânsito pode servir de instrumento poderoso de desenvolvimento da pessoa humana na busca do exercício da cidadania e preservação à vida.\vskip0.3cm

A Constituição do Estado do Pará, promulgada em 05 de outubro de 1989, inspirada nos princípios constitucionais da República Federativa do Brasil, no seu Capítulo II, que trata da Competência do Estado, já previa a necessidade da implantação da educação de trânsito, conforme texto constitucional onde no seu mencina no Art. 17 - É competência comum do Estado e dos Municípios, com a União: XII - estabelecer e implantar política de educação para a segurança do trânsito. \vskip0.3cm

Assim, a Educação de Trânsito, prevista na Constituição Estadual, ocupa o seu lugar na Lei maior que representa a vontade do povo, permitindo atribuir à mesma um grau de importância tamanha, devido esta modalidade de educação aparecer como uma responsabilidade ao lado da saúde, cultura, educação, ciência, meio ambiente, entre outros, permitindo compreender a importância da mesma, principalmente, por tratar da segurança no trânsito da pessoas que vivem e transitam pelo vasto território paraense. 


\section{Educação de Trânsito via CTB}

Na década de 1910, começa a elaboração das legislações, com a finalidade de disciplinar o trânsito e estimular a criação de estradas para ligar as regiões brasileiras através do uso de veículos automotores como meio de transporte. Os códigos de trânsito brasileiro existentes, além de buscarem disciplinar o trânsito, atuavam nas questões atinentes à atividade policial, sinalização e segurança no trânsito, produzindo os seus efeitos até 1996, quando, em substituição ao Código Nacional de Trânsito, foi promulgado o Código de Trânsito Brasileiro (CTB) atualmente em vigor.\vskip0.3cm

O novo CTB foi criado pela Lei nº 9.503 e sancionada pela Presidência da República em 23 do mês de setembro do ano de 1997. E está definido no seu Art. 1º, § 2º, que o “[...] trânsito, em condições seguras, é um direito de todos e dever dos órgãos e entidades componentes do Sistema Nacional de Trânsito, a estes cabendo, no âmbito das respectivas competências, adotar as medidas destinadas a assegurar esse direito” CTB (BRASIL, 1997, p.1).\vskip0.3cm

Pelo exposto, sabe-se que a lei trouxe muitas inovações para a regulamentação do trânsito no Brasil, instituída por leis e decretos, além de resoluções dos conselhos competentes. Assim, o CTB tem a finalidade de regular o sistema nacional de transito, tendo em vista as normas gerais de circulação e conduta para motoristas e pedestres, além de estabelecer a obrigatoriedade de educação para o trânsito.\vskip0.3cm


O CTB é um documento extenso, constituído inicialmente com 341 artigos e seus respectivos parágrafos, organizados em 20 capítulos que definem atribuições das diversas autoridades e órgãos ligados ao trânsito do Brasil. 



\section{Educação de Trânsito via BNCC}